\section{Einleitung}
Dieser Projektantrag informiert die an der Semestarbeit ''Vacation Planner'' von Raffael Schmid beteiligten Parteien \"uber den Projektrahmen, die Projektziele und den Projektumfang. Der Projektantrag soll in erster Linie einen \"Uberblick von ''Vacation Planner'' erm\"oglichen.

\section{Ausgangslage}
\subsection{Problemstellung}
In der heutigen Zeit gibt es viele Teams und Kleinunternehmen, die Ihre Ressourcenplanung noch immer auf Basis von Papier und Excel durchf\"uhren. Dies f\"angt bei einfachen Problemstellungen wie Ferienplanung an, und h\"ort bei Komplexen Szenarien wie die Planung und Organisation von Ressourcen in Schichtbetrieben auf.

\subsection{Vacation-Planner}
Die Idee f\"ur einen Ressourcen-Planer stammt aus dem Umfeld des betreuenden Dozenten Beat Seeliger. Heruntergebrochen auf den Umfang einer Semesterarbeit von rund 120 Stunden war diese Problemstellung allerdings zu Umfangreich und man entschied sich, einen Teil der Problemstellung zu bearbeiten. Daraus ergab sich die Projektidee ''Ferienplaner''.

\section{Projektzielsetzung}
Der Ferienplaner wird auf der Basis der Java-Plattform erstellt. Ziel der Arbeit ist es zus\"atzlich, eine Evaluation des auf der Sprache Scala\footnote{http://www.scala-lang.org} basierenden Webframeworks (Lift) durchzuf\"uhren, um danach entscheiden zu k\"onnen,  wie das Potential dieses Frameworks f\"ur Projekte in diesem Rahmen ist. Ansonsten werden bereits bew\"ahrte Komponenten verwendet.


\section{Projektvarianten}
\subsection{Backend}
Im Bereich der Softwareentwicklung auf der Java-Plattform gibt es unterschiedliche Frameworks, und immer wieder stossen neue Ideen und Ans\"atze hervor. Aktuell wird die Sprache Scala ''gehyped''. Darauf basierend existiert ein Webframework, welches viele neue Ideen und Konzepte integriert. Ein Teil der Semesterarbeit soll sich auch damit besch\"aftigen, ob sich dieses Framework und die Sprache Java momentan f\"ur eine weitere Betrachtung lohnen, oder ob die Zeit (noch) nicht reif daf\"ur ist. Es werden in etwa ein viertel der zur Verf\"ugung stehenden Zeit in die Evaluation investiert.
\subsubsection{Scala, Lift-Framework}
Der erste Milestone besch\"aftig damit, ob auf der Basis von Liftweb fortgefahren wird, oder ob auf ein anderes Framework ausgewichen wird.

\subsubsection{Groovy, Grails}
Groovy und Grails sind bereits bew\"ahrtere Partner, und werden verwendet, falls es f\"urs Scala-Framework ein ''nogo'' gibt.

\subsection{Frontend}
Im Bereich des Frontends gibt es zwei verschiedene M\"oglichkeiten, f\"ur die eine Evaluation sowie einen Variantenentscheid durchgef\"uhrt werden muss.
\subsubsection{HTML, Javascript, CSS}
Der klassische Weg mit HTML, Javascript und CSS ist f\"ur die Entwicklung von Administrations-Oberfl\"achen weniger effizient als RIA\footnote{Rich Internet Applications} Produkte. Ausschlaggebend f\"ur einen Entscheid ist die Evaluation von Komponenten zur Darstellung von Gantt-Charts\footnote{Charts zur darstellung von Ressourcen, usw.} respektive Scheduling-Charts.

\subsubsection{Flex}
Flex ist ein Framework von Adobe und bietet vielseitige Unterst\"utzung bei der Erstellung von RIA Oberfl\"achen. Dabei wird Flex-Code in Actionscript-Code umgewandelt und dann f\"ur den Flash-Player kompiliert. Die Tatsache, dass f\"ur die Anwendung nur ein Browser mit installiertem Flash-Plugin notwendig ist und deshalb keine Cross-Browser Probleme\footnote{Probleme die im Zusammenhang mit den Unterschiedlichen Browsern auftreten} auftreten, macht die Entwicklung von Anwendungen sehr effizient.

\section{Projektrandbedingungen}
Das Projekt ''Vacation Planner'' wird im Rahmen der Semesterarbeit an der Fachhochschule f\"ur Technik in Z\"urich durchgef\"uhrt und unterliegt deren \"ublichen Bedingungen.

\section{Wirtschaftlichkeit}
\subsection{Theoretische Kosten}
\begin{tabular}[ht]{l|ccc}
  \hline
  Posten & Stunden & Preis & Total\\
  \hline
  Personalaufwand & 1 x 140h & 125 SFr/h &17'500 SFr \\
  Projektbetreuung & 1 x 10h & 200 SFr/h & 2'000 SFr \\
  evt. Lizenzkosten &  &  & 2'000SFr \\
  \hline
  Total & & & 21'500 SFr.\\
  \hline
\end{tabular}

\subsection{Nutzen}
Ziel des End-Produktes ist es, das Produkt im Rahmen einer SaaS \footnote{Software as a Service} zu Verkaufen. Dar\"uber k\"onnte man es via eines Service Agreements verkaufen.

\section{Konsequenzen}
\subsection{Bei Realisierung}
Dieses Projekt ist mit sehr geringen Risiken verbunden. Da weder finaniziell noch personell ein grosser Aufwand geplant ist. Das gr\"osste Risiko beinhaltet die Problematik, in der geplanten Zeit nicht \"uber die Runden zu kommen.

\subsection{Bei Nichtrealisierung}
Nichtrealisierung ist nicht m\"oglich. Da die Konsequenzen einer schlechten Note nicht tragbar sind;)

\subsection{Versp\"atete Realisierung}
Versp\"atete Realisierung ist tendenziell auch keine Option.;)

\section{Projektantrag}
Unter den in diesem Bericht geschilderten Gegebenheiten geht die Projektgruppe davon aus, dass die Erstellung dieses Prototypen f\"ur die Ferienplanung, auch unter Ber\"ucksichtigung des Zieles ''Aufbau von Knowhow und Erfahrungen'', sinnvoll und n\"utzlich ist. Die Projektgruppe ersucht darum, dem Projektantrag und damit dem Start des Projektes zuzustimmen.