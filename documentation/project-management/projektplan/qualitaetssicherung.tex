\chapter{Qualit\"atssicherung}
Im folgenden werden die Qualit\"atssichernden Massnahmen in konstruktive, analytische und organisatorische Massnahmen eingeteilt. Die gr\"osse des Projektes l\"asst es nicht zu, allzu viel im Bereich Qualit\"atssicherung zu tun. 

\section{Konstruktive Massnahmen}
\subsection{UML}
Die Modellierung wird mit der Unified Modeling Language durchgef\"uhrt. 

\section{Analytische Massnahmen}
\subsection{Continuous Integration, Testing}
W\"ahrend der Entwicklungszeit wird als Continuous Integration Server Hudson\footnote{http://hudson-ci.org} eingesetzt. Hudson checkt t\"aglich den Code aus dem Git\footnote{Git ist eine Software zur Source-Code Kontrolle: http://git-scm.com}-Repository aus und f\"uhrt einen Build der Applikation durch, um anschliessend die Unit-Tests, Functional-Tests und Integration-Tests durchlaufen zu lassen. Zur statischen Sourcecode Analyse stellt Hudson eine Reihe von Plugins  f\"ur bereits bestehende Tools zur Verf\"ugung:
\begin{itemize}
\item FINDBUGS: FindBugs ist ein Open Source Programm urspr\"unglich von Bill Pugh und David Hovemeyer entwickelt, welches in Java-Code nach Fehlermustern sucht. Solche Fehlermuster deuten meist auf tats�chliche Fehler hin. Das Programm wurde von der University of Maryland  aus initiiert, mittlerweile umfasst das Entwicklerteam fast ein Dutzend Personen
\item PMD: Die Fehler, die PMD findet, sind typischerweise keine echten Fehler, sondern eher ineffizienter Code, d. h. die Software wird in der Regel trotzdem korrekt ausgef\"uhrt, wenn die Fehler nicht korrigiert werden. PMD findet auf Basis von statischen Regeln potentielle Probleme wie beispielsweise m\"ogliche Bugs, toter Code, \"Uberkomplizierte Ausdr\"ucke, Suboptimaler Code, Klassen mit hoher zyklomatischer Komplexit\"at
\item CHECKSTYLE: Das Plugin Checkstyle erm\"oglicht eine automatische \"Uberpr\"ufung der Einhaltung von Coding Conventions bei der Erstellung von Code. 
\end{itemize}

Die Reports der verschiedenen Plugins k\"onnen auf einer Webseite angezeigt werden. Sofern die Tests (Unit-, Functional-, Integration-Tests) nicht erfolgreich durchlaufen werden, wird ein Mail an die Entwickler-Crew (in diesem Fall an den Studenten) versendet.


\section{Organisatorische Massnahmen}
Organisatorische Massnahmen wie Audits, Programmier- und Dokumentationsrichtlinien, Konfigurationsmanagement werden keine definiert. Das Glossar dient zum Verst\"andnis der Dokumentation.