\section{Teilnehmer}
Beat Seeliger, Raffael Schmid

\section{Einleitung}
Der Student Raffael Schmid hat eine kurze Pr\"asentation der Aufgabenstellung vorbereitet. Da 
der Dozent Beat Seeliger \"uber die Aufgabe respektive den Umfang der Arbeit gut im Bild war, wurde auf diese allerdings verzichtet.

\section{Ablauf}
\begin{enumerate} 

\item Begr\"ussung
\item Korrektur Aufgabenstellung
\item Hinweise zum Ablauf
\item Hinweise zu den Beurteilungskriterien

\end{enumerate}

\section {Punkt 2: Korrektur Aufgabenstellung}
Der Dozent m\"ochte ein zus\"atzliches Ziel in die Aufgabenstellung aufnehmen:
\begin{itemize}
\item Analyse der Unterst\"utzung des Lift Frameworks f\"ur das automatisierte Testen (Unit Tests, Functional Tests)
\end{itemize}

\section {Punkt 3: Hinweise zum Ablauf}
Der Dozent weist ausdr\"ucklich darauf hin, dass das Design Review der letzte Termin ist, um die formulierten Ziele zu \"andern, dies allerdings nicht ohne triftigen Grund.

\section{Punkt 4: Hinweise zu den Beurteilungskriterien}
Folgende Punkte wurden im Speziellen hervorgehoben:
\subsection{Wissenschaftliche Arbeitstechnik}
Bei Problemen respektive Entscheiden w\"ahrend der Umsetzung sollen die verschiedenen M\"oglichkeiten aufgezeigt werden, Vor- respektive Nachteile dieser und der gew\"ahlte Weg dokumentiert werden.
\subsection{Dokumentation}
Bei der Dokumentation wird zus\"atzlich zum Inhalt das formale Erscheinungsbild bewertet - deshalb wird Latex anstelle von konventionellen Textverarbeitungsprogrammen seitens Dozenten empfohlen.
