\section{Einleitung}
Auf der Basis von Lift und der Programmiersprache Scala wurde ein Prototyp zur Planung von Ferien innerhalb Gruppen oder Teams entwickelt. Ziel dieser Arbeit war nebst dem fertigen Prototypen auch die Einsch\"atzung dieses relativ neuen Frameworks und der Sprache Scala. Das daf\"ur notwendige Knowhow wurde teils zuvor, teils w\"ahrend der Semesterarbeit erarbeitet. 

\section{Analyse der Aufgabenstellung}
Der Prototyp zur Ferienplanung soll im wesentlichen die folgenden Use Cases abdecken:
\begin{itemize}
\item Benutzeradministration von Team-Membern und -Ownern
\item Erfassung von Teams und zuteilung von Mitarbeitern
\item Eingabe und Best\"atigung von Ferien f\"ur Mitarbeiter.
\end{itemize}

W\"ahrend der Designphase sollen Rollenkonzept, Prozesse, Navigation und das Relationale Model definiert werden.

\section{Grundlage}
Scala entstand im Wesentlichen auch mit dem Ziel der Erneuerung der Sprache Java. Dies war aufgrund der Anforderungen an die R\"uckw\"artskompatibilit\"at nicht im erw\"unschten Masse m\"oglich. 
Viele der Konzepte innerhalb Scala sind auch bereits aus anderen Sprachen bekannt, nebst dem grossen Einfluss aus der Java-Ecke gab es auch unterschiedliche andere Sprachen, die in Scala ihre Spuren hinterliessen: Erlang ist erkennbar durch die Actors, Standard ML durch das Typsystem, Haskell, Smalltalk, etc. Es handelt sich also nicht um einen Klon auf die JVM einer bereits existierenden Sprache wie bei Jython (Python), JRuby (Ruby), Clojure (Lisp). 

\subsection{Eigenschaften}
\subsubsection{Scala}
Bei Scala handelt es sich nicht um eine rein Funktionale Sprache, allerdings erf\"ullt sie viele der daf\"ur notwendigen Bedingungen:
\begin{itemize}
\item Anonyme Definition von Funktionen
\item Funktionen werden wie Daten behandelt. D.h. es gibt in einer statisch typisierten Sprache einen passenden Datentyp zu jeder Funktion.
\item Funktionen k\"onnen als Parameter und/oder R\"uckgabewerte von anderen Funktionen sein.
\end{itemize}

Weitere Eigenschaften der Sprache Scala sind: objektorientiert, statisch Typisiert.
\subsubsection{Lift}
Lift ist ein Webframework basierend auf der Sprache Scala, viele der Funktionalit\"aten die Lift zur Verf\"ugung stellt basieren deshalb auf dessen Sprachfeatures (Bsp: XML-Datentypen, etc.). Grunds\"atzlich sind viele Ans\"atze bei Lift relativ neu und innovativ und m\"oglicherweise in gewissen Bereichen der Zeit etwas voraus. F\"ur gew\"ohnliche Webseiten ist es m.E. nicht m\"oglich, in Sachen Entwicklungsgeschwindigkeit mit bereits etablierten Frameworks wie Rails oder Grails mitzuhalten. Hinzu kommt, dass die komplexit\"at der darunterliegenden Sprache die Verbreitung des Frameworks nicht f\"ordert. 

\subsection{Design und Konzeption}
Bevor der Prototyp in Angriff genommen wurde, wurden basierend auf den Use Cases Rollenkonzept, Navigationskonzept und die Prozesse definiert. Das Relationale Modell der Datenbank wurde anhand dieser Konzepte definiert. 

\section{Implementation Prototyp}
Als Basis f\"ur den Prototypen wurden folgende Frameworks verwendet:
\begin{itemize}
\item \textbf{Maven: } Als Basis f\"ur Lift-Projekte kann sowohl Maven als auch SBT\footnote{Simple Build Tool} verwendet werden. Die gr\"osse Verbreitung von Maven f\"uhrt zu einem reichhaltigen \"Okosystem an Plugins. So konnte beispielsweise das Projekt mittels Plugin auf die Stax-Plattform\footnote{Stax basiert auf der Amazon Elastic Cloud} deployt werden.
\item \textbf{Flex: }Ein Teil der Funktionalit\"at im Frontend wurden mittels Flex entwickelt. Der Zugriff auf das Backend findet via RESTful Webservices statt.
\item \textbf{JPA: } Als Basis f\"ur die Persistenz wurde aus verschiedenen Gr\"unden (Reifegrad, Dokumentation) JPA 2.0 den Lift-internen Bibliotheken Mapper und Record vorgezogen.
\end{itemize}

