\chapter{Informationen}
\section{Inhalt des Datentr\"agers}
\begin{longtable}{|p{6cm}|p{8cm}|}
\hline
\textbf{Pfad}&\textbf{Beschreibung}\\\hline
  / & \\\hline
  /documentation & Beinhaltet alle Teile der Dokumentation: Semesterarbeit, Protokolle von Design-Review und Kickoff und die Aufgabenstellung.\\\hline
  /workspace-client & Workspace f\"ur den Flash-Builder, mittels welchem die Teile Teamadministration, Kalender f\"ur die Ferienplanung in Flex implementiert wurden. \\\hline
  /workspace-server & Hier befinden sich die Teile f\"ur das Backend und die Seiten, welche in HTML, CSS implementiert wurden.\\\hline
 /workspace-server/intellij-project & Projektdateien f\"ur die Entwicklungsumgebung\\\hline
 /workspace-server/plannr & Hier befinden sich die Source-Dateien f\"ur das Backend. Es beinhaltet zwei Projekte, ``plannr-test'' in welchem sich wenige Klassen zu Testzwecken befindet und ``spa'', in welchem sich die ganze Backend-Applikation befindet. \\\hline
\end{longtable}
  

\section{Diverses}
\subsubsection{Git Repository}
Das Git Repository der Semesterarbeit befindet sich unter \cite{gitRepo}.

\subsubsection{Url}
Die Applikation befindet sich auf Stax und kann unter \cite{Plannr} gefunden werden.\footnote{Normalerweise dauert der Seitenaufbau beim ersten Aufruf relativ lange, da sich Applikationen auf Stax nach einem Tag wieder in den Hibernate Modus versetzen.}