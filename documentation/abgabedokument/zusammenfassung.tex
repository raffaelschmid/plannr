\chapter*{Zusammenfassung}
Ziel dieser Semesterarbeit ist es, die M\"oglichkeiten und das Potential der Programmiersprache Scala respektive des Webframeworks Lift zu erforschen und das notwendige Knowhow zu erarbeiten. In erster Linie wurden Funktionalit\"aten wie die Persistenz, Internationalisierung und Support f\"ur RESTful Webservices untersucht. Daneben ging es aber auch um die Analyse von nichtfunktionalen Eigenschaften wie Architektur, Erweiterbarkeit, Deployment und Testbarkeit.

Zum Erreichen dieses Ziels wird auf der Basis von Lift und Scala eine Webapplikation zur Ferienplanung erstellt. Diese war urspr\"unglich als Basis f\"ur eine Software zur Ressourcenplanung gedacht - dient aber in erster Linie vorerst als ''Spielwiese'' um verschiedene Anforderungen der Zielsoftware zu diskutieren.

Die resultierende Webapplikation wurde mittels dem Webframework Lift als Backend implementiert, das Frontend besteht aus einem Flex-Client \footnote{Flex ist ein Framework von Adobe mittels welchem man mit relativ geringem Zeitaufwand Webclients erstellen kann. http://www.adobe.com/de/products/flex} der via REST Schnittstelle auf die Services im Hintergrund zugreifft. Zur persistierung wurde die Java Persistence API durch die Implementation Hibernate verwendet und als Programmiersprache wurde Scala verwendet. Die Applikation l\"auft Produktiv in der STAX\footnote{http://www.stax.net} Cloud.