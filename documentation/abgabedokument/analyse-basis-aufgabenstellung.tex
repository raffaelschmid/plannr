
\chapter{Analyse der Arbeit auf der Basis der Aufgabenstellung}
\section{Zielerreichung}
\subsection{Prototyp}
\subsubsection{Entwicklung}
\subsubsection{Deployment}

\section{Optionale Ziele}
Die von mir in der Aufgabe definierten optionalen Ziele sind heutzutage eben genau die Kriterien, welche in der Menge aller Web-Applikationen die Spreu vom Weizen trennt. Deshalb w\"are es interessant gewesen, wie das Lift Framework und Scala in diesen Bereichen abschliesst. Auf das Performance Verhalten habe ich meine Applikation nicht getestet, und auch die anderen drei Punkte habe ich im Prototypen nicht umgesetzt. 
\begin{itemize}
\item  \textbf{Internationalisierung: } Internationalisierung l\"asst sich in Lift wie in anderen Java-Frameworks auf der Basis von Properties einfach l\"osen. Siehe Abschnitt ``\ref{lift:internationalisierung} \titleref{lift:internationalisierung}''
\item \textbf{Search Engine Optimization: }Aus meiner Sicht haben die meisten SEO-Massnahmen mit den folgenden, unabh\"angig vom verwendeten Framework, aufgelisteten Punkten einer Webseite zu tun und sind deshalb f\"ur die Auswahl dessen nicht besonders relevant:
\begin{itemize}
\item Inhalt
\item Meta-Daten
\item Mehrsprachigkeit
\item Sprechende Links
\end{itemize}
\item \textbf{Usability: } Auch hier geht es m.E. in erster Linie um nicht Framework-relevante Punkte wie:
\begin{itemize}
\item Design und Inhalt der Seite
\item Navigationsstruktur
\item Fehlertoleranz
\end{itemize}
\end{itemize}

