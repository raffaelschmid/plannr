\chapter*{Einleitung}
W\"ahrend vielen Stunden habe ich mich mit Scala, dem Lift Webframework, dem Design eines Prototypen zur Ferienplanung, der entsprechenden Implementation und der Dokumentation besch\"aftigt. F\"ur mich waren die Technologien relativ neu, ich kannte sie nur aus der Theorie und wie sehr oft in der Informatik, aller Anfang war schwer.  Die ersten Versuche waren zwar wegbereitend, aber m\"uhsam. Die Einarbeitung in die verschiedenen Persistenz Frameworks war sehr aufw\"andig und ich war unmittelbar davor, auf ein mir bekanntes Webframework zu wechseln. Ich h\"atte dann das Ziel verfehlt, mir am Ende ein Bild von Scala und Lift machen zu k\"onnen. Die Prozesse der Applikation waren mir zu diesem Zeitpunkt schon relativ klar, trotzdem ging es jetzt an die Definition dieser,  ans erstellen eines passenden Rollenkonzepts, Navigation und basierend auf diesen Vorgaben zu Design und anschliessend Implementation des Dom\"anenmodells mit JPA. Nun war es relativ schnell m\"oglich, Objekte in der Datenbank zu persistieren und dem lang ersehnten Flow stand nichts mehr im Wege. Der Planung lief ich zu diesem Zeitpunkt leider schon etwas hinterher, denn im Allgemeinen habe ich mir unter ``fast to build''\cite{liftweb} einen einfacheren Einstand vorgestellt. Trotzdem, gegen Ende schwanden die Wolken am Horizont, es entstand eine gewisse Hingabe und ich \"uberlege mir, wo ich Scala oder Lift als n\"achstes einsetzen k\"onnte. Hoffentlich hilft die eine oder andere Feststellung auch weiteren Leuten oder ich kann jemanden f\"ur Scala oder Lift begeistern.

Das Dokument der Semesterarbeit ist in drei Teile gegliedert. Teil eins mit der Aufgabenstellung, Teil zwei mit deren Analyse, den zur Implementation ben\"otigten Grundlagen, Design und Konzeption, Implementation sowie der Beschreibung von Entwicklungs- und Testumgebung. Anschliessend folgt im dritten Teil der R\"uckblick, hier wird die Applikation anhand der Vorgaben bewertet und ein Fazit \"uber die verwendeten Technologien gezogen. \newline\newline
Danke f\"urs Interesse\newline
Raffael Schmid