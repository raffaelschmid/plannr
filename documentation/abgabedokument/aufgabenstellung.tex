\chapter{Aufgabenstellung}\label{aufgabenstellung}
\section{Ausgangslage}
In vielen Firmen, welche ohne ERP Software auskommen, wird die Planung von Ressourcen manuell mit der Hilfe verschiedenster Standardsoftware (Excel, Access) oder Papier gemacht. Was in vielen Projekten, Unternehmen fehlt, ist die Software zur Planung von Ressourcen. Als Grundlage dazu soll ein webbasierter Ferienplaner implementiert werden, der sp\"ater sukzessive zu einer Gesamtl\"osung erweiter werden kann. Zur Implementierung dieses Prototypen wird das Lift Webframework verwendet. Lift \footnote{http://liftweb.net} ist ein Framework auf der Basis von Scala (eine Programmiersprache die mitunter an der Ecole Polytechnique Fédérale de Lausanne entwickelt wurde) und verinnerlicht auch deshalb in vielen Bereichen v\"ollig neue Konzepte. Der Prototyp soll auch die M\"oglichkeiten der doch eher neueren Bibliothek transparent darstellen. Ich m\"ochte darauf hinweisen, dass ich mir zus\"atzlich zum Prototypen das Knowhow im Bereich Scala und Lift erarbeiten muss. Die Arbeit soll es mir erm\"oglichen, eine Aussage \"uber das Potential von Lift machen zu k\"onnen. Scala hat, wenn man sich auf Magazine oder verschiedenster Internetseiten wie zum Beispiel den Tiobe-Index\footnote{http://www.tiobe.com/index.php/content/paperinfo/tpci/index.html} bezieht, den Durchbruch ja schon fast geschafft. 

\section{Ziel der Arbeit}
Als Vorarbeit zur Konzeption und Entwicklung dieses Prototypen muss das Knowhow im Bereich Lift respektive Scala erarbeitet werden. Bei diesem Prozess stehe ich zwar nicht mehr am Anfang, ich werde jedoch trotzdem zus\"atzlich Zeit zum Aufbau meines Wissens ben\"otigen. Darauf aufbauend werden die Requirements der Applikation definiert und entsprechende Konzepte erstellt (User, Rollen, Prozesse). Aufgrund dieser Requirements werden Recherchen durchgef\"uhrt, um herauszufinden, ob Konzepte oder Teile bestehender L\"osungen \"ubernommen werden k\"onnen. Anschliessend werden die Requirements des Prototypen umgesetzt. 

\subsection{Optionale Ziele}
Ich erwarte, dass diese Zielsetzung den Rahmen einer Semesterarbeit bereits deckt, f\"ur den Fall dass ich noch Zeit finde, fasse ich optional noch folgende Punkte ins Auge: 

\begin{enumerate}
	\item Setup von Test- und Produktiver Umgebung
	\item Performance Testing
	\item Internationalization 
	\item Search Engine Optimization
	\item Usability
\end{enumerate}

\section{Aufgabenstellung}
Erarbeitung einer Wissensbasis im Bereich Lift und Scala, um die Konzeption und Implementation des Prototypen zu erm\"oglichen. Setup der Entwicklungs-Infrastruktur. Dies beinhaltet das Projektsetup mit Maven, Versionskontrolle mit Git oder Subversion, Einrichten der Entwicklungsumgebung, Aufsetzen Infrastrukur f\"ur automatisierte Testl\"aufe, Entwicklungsserver, allerdings werde ich auf den Einsatz einer Continuous Integration Software verzichten. W\"ahrend der Konzeption werden Navigations- und Rollenkonzepte erstellt sowie die verschiedenen Prozesse definiert. Implementation des Prototypen beinhaltet unter anderem Folgendes:

Aufbau des Domain Modells - Implementation der Persistenz-Schicht - Security: Registrierung, Login, ... - Navigation - Mail-Versand - Evaluierung eines geeigneten, auf Javascript basierendem Kalender, um die Pers\"onliche Ferien\"ubersicht sowie die Team\"ubersicht darstellen zu k\"onnen.

\section{Erwartetes Resultat}
Dokumentation der Semesterarbeit beinhaltet unter anderem folgende Teile: 
\begin{itemize}
\item Konzepte
	\begin{itemize}
		\item Navigationskonzept
		\item Rollenkonzept
		\item Prozesse Dokumentation
	\end{itemize}	
	\item Implementationsdetails
	\item Entscheide Software
	\item Lauff\"ahige Software als Maven-Projekt
\end{itemize}



