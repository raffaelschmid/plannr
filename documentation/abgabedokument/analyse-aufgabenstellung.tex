\chapter{Analyse der Aufgabenstellung}\label{analyse-aufgabenstellung}
\section{Idee und Ziele der Arbeit}
Grunds\"atzliche Idee der vorgelegten Semesterarbeit war, anhand der Entwicklung eines Prototypen Erfahrungen mit Scala und dem Lift Framework zu machen, und um heraus zu finden wie das Potential beider Technologien ist. Zu Beginn steht nat\"urlich die Erarbeitung des Knowhows in beiden Bereichen Scala (siehe Abschnitt \ref{einarbeitung:scala} \titleref{einarbeitung:scala}) , Lift (siehe Abschnitt ''\ref{einarbeitung:lift} \titleref{einarbeitung:lift}'' im Zentrum. In der Folge sind die aus der Aufgabenstellung entstehenden Ziele definiert.

\subsection{Vorbereitung: Erarbeitung des Basiswissens}
Nebst dem Projekt-Setup und der Einarbeitung in Scala (siehe dazu Abschnitt ``\ref{einarbeitung:scala} \titleref{einarbeitung:scala}'')  besteht ein wesentlicher Bestandteil dieses Punktes auch darin, herauszufinden wie die folgenden Problemstellungen, die sich bei der Idee des Ferienplaners nicht wesentlich von anderen Gesch\"aftsideen unterscheiden, mit Lift umsetzen lassen:
	\begin{itemize}
		\item Authentifizierung, Authorisierung
		\item Persistenz respektive Objekt-Relationales Mapping
		\item Internationalisierung
		\item Architektur von Applikationen
		\item Testbarkeit
	\end{itemize}
Die Ergebnisse sind im Abschnitt  ``\ref{einarbeitung:lift} \titleref{einarbeitung:lift}'' dargestellt.

\subsection{Design}
Nachdem die Einarbeitung abgeschlossen ist, kann mit dem Design der Arbeit begonnen werden. Dieser Punkt beinhaltet die Definition der Use Cases, der Prozesse des Navigationskonzepts, des Rollenkonzepts und das Design der Datenbank.

\subsection{Technische Umsetzung}
Im Anschluss an die Design-Phase wird der Prototyp umgesetzt. Die Details zur Umsetzung befinden sich im Abschnitt ``\ref{implementation} \titleref{implementation}''.


\section{Lieferumfang der Semesterarbeit}
Nebst dem \textbf{Prototypen} besteht die Abgabe aus \textbf{Dokumentation}, die folgende Teile beinhalten soll:
\begin{itemize}
	\item Konzepte
	\begin{itemize}
		\item Navigationskonzept befindet sich im Abschnitt \ref{konzept:navigation} \titleref{konzept:navigation}.
		\item Rollenkonzept befindet sich im Abschnitt ``\ref{konzept:rollen} \titleref{konzept:rollen}''.
		\item Die Definitionen der Prozesse befinden sich im Abschnitt ``\ref{konzept:prozesse} \titleref{konzept:prozesse}''
	\end{itemize}
	\item Implementationsdetails befinden sich im Abschnitt ``\ref{implementation} \titleref{implementation}''
\end{itemize}

Die Definition des Wortes Prototyp ist bekanntlich ein bisschen schwammig. In meinem Sinne sollte der Prototyp die folgenden Punkte erf\"ullen:

\begin{enumerate}
\item \textbf{Potential} - Anhand der Implementation des Prototypen solle eine qualifizierte Aussage dar\"uber gemacht werden, wie viel Potential in Scala und Lift steckt. 

\item \textbf{Erkenntnisse} - In den tangierten Bereichen (Persistenz, Webservices, usw.) sollen Ans\"atze von Best Practices erarbeitet werden respektive Aussagen \"uber die Vor- und Nachteile von verschiedenen Technologien gemacht werden k\"onnen. Dies betrifft auch die Bereiche Entwicklungsumgebung, Build-Tools, usw.

\item \textbf{Abdeckung des Funktionsumfangs} - Ich werde versuchen, einen m\"oglichst grossen Funktionsumfang der Applikation zu implementieren.
\end{enumerate}






